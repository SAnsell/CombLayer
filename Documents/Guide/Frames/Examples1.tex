
\section{Example 1}

\subsection{Purpose: Creating the initial world}

This example is the first stage on the process of building a full model.
The only purpose of this example is to create an unpopulated world (a large
void sphere) and generate an executable that can output a valid MCNP(X),
FLUKA or PHITS output.

\subsection{Creating required files}

Assuming that we are going to build a new project that does not
currently exist and for this case, we will call it {\it example}
\footnote{The completed version of this test can be obtained by
checking out the example1 branch of CombLayer}.

\subsubsection{Create a new main() function}

Move into the Main directory:

Choose an existing project and copy it  --  in this case
use the xrayHut.cxx file int example.cxx

\subsection{Description of the main file}


Consider main file {\it Main/example.cxx}:
\begin{itemize}
\item{{\bf includes [lines 22-58]} : This are all the standard include. They allows the construction of the makeExample class}
\item{{\bf ELog construction [lines 60-67]} : This is standard static constructors for the controlled streams (they are like std::cout but have more control). Don't touch! }

\item{{\bf activateLogging [lines 75]} : This activates the logging
  system, that will give output with the class and method that output
  came from. It also gives a full calling structure of exceptions.}
  
\item{{\bf createInputs [lines 86]} : This function generates a template
  of input flags that will be recognised by CombLayer.
  The call to createInputs creates the general input flag (e.g.
  --FLUKA) they are general, but if you need any special calls for
  your code, then create a new funtion here (e.g. createExampleInputs)
  and typically that function will call createInputs adding
  the special extra parts you need.}


\item{{\bf activateLogging [lines 75]} : This activates the logging
  system, that will give output with the class and method that output
  came from. It also gives a full calling structure of exceptions.}
  

\item{{\bf setMaterialDataBase [lines 77]} : CombLayer provides a list
  of named (not only numbers) materials that the user can use. This
is where they are populated.}

\item{{\bf createSimulation [lines 78]} : This does NOT create the model
  but it produces the empty simulation of the correct type e.g. MCNP or FLUKA.
  This is controlled by the inputs on the command line (e.g --FLUKA).}

\item{{\bf exampleVariables [lines 93]} : This is a user written function.
  It populates all the variables for the model that you are building. Every variables should be defined by this function.}

\item{{\bf makeExample BOj [lines 95-96]} :
  This creates the class which is going to construct the whole object
  model. Doesn't matter if this is simple (as this case) or super complex,
  just one call. }

\item{{\bf buildFullSimulation [lines 100-104]} :
  This does the construction of the output file (based normally on inputs
  etc).}

\item{{\bf Catch [lines 100-104]} :
  This does the construction of the output file (based normally on inputs
  etc).}
  
  


\subsection{Changes for the main file}

Edit example.cxx and make standard changes because we don't want name overlap.
In the {\it #include}'s change

\begin{itemize}
\item{ {\it #include ``xrayHutVariables.h''} $\rightarrow$
  {\it \#include ``exampleVariables.h''} }

\item{ {\it #include ``makeHutch.h''} $\rightarrow$
  {\it \#include ``makeExample.h''} }

\item{{\it setVariable::xrayHutVariables(SimPtr->getDataBase());}
  $\rightarrow$ {\it setVariable::exampleVariables(SimPtr->getDataBase());}

\item{
  {\it  exampleSystem::makeXrayHut BObj; }
  $\rightarrow$
  {\it  exampleSystem::makeExample BObj; }}
\end{itemize}

