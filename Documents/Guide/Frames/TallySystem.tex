
Tallies are the fundamental reason for running MCNPX. However, the
manor in which MCNPX specifies tallies is not compatible with a
vairiable defined model because in most cases the required tally is relative 
to an object whose position is unknown. 

This problem has been addressed by allowing most tallies to use the 
FixedComp link system. 

\subsection{Tally System}

The tally system is accessed either by a simple command line menu
system, or via an XML file. The command line help system is very
primative but can remind the user of the bais

\subsection{Point Tally}
\label{PointTally}

Point tallies are fundamentally a 3D vector in space. In CombLayer,
there are three levels of position available: (a) Real MCNP(X) output
position, (b) CombLayer master origin position before master rotation,
and (c) relative position to an object. Both (a) and (c) are well
supported, however, to do option (b) there needs to be some real care
with the layout of the calling sequence in the main() function. The
fullBuild.cxx example is a suitable option to follow, but checking
will be needed.

\subsubsection{Free Point tally}

The simplest way to put a point into CombLayer is to use a free point.
\begin{bash}
./prog -T point free 'Vec3D(300.0,10.0,5.0)' Output
\end{bash}

This creates a point tally at (300,10,5) in the final output using
neutron tallies with the default energy and time binning system.

\subsection{Fluka Tally System}

\subsubsection{Surface Tally}

The fluka surface tally is accessed via -T surface.

The 
