\section{Installation}

CombLayer is primarily designed for the Linux platform and requires a
\CC compiler that supports {\CC}11 or later. The code is available at
\href{https://github.com/SAnsell/CombLayer}{https://github.com/SAnsell/CombLayer},
either as a downloadable
\href{https://github.com/SAnsell/CombLayer/archive/master.zip}{zip
  file} or by cloning/pulling the Git repository:
\begin{bash}
  git clone https://github.com/SAnsell/CombLayer.git
\end{bash}

\subsection{Requirments}

CombLayer requires the GNU Scientific
Library~(\href{https://www.gnu.org/software/gsl}{GSL}),
\href{https://www.boost.org}{boost::regex}, as well as the
\href{https://github.com/fmtlib/fmt}{\{fmt\}} and STL libraries
provided by your \CC compiler. The build process is managed with
\href{https://cmake.org}{CMake}, and the class reference documentation
is generated using \href{https://www.doxygen.nl}{Doxygen}.
Both {\tt gcc} and {\tt clang} compilers are supported.

\subsection{Basic build method}

CombLayer can be built either in the source code directory or in a separate build directory. If built in the source directory, the command is

\begin{bash}
  cmake ./
  make
\end{bash}

or for a separate build directory

\begin{bash}
  cmake -B/path/to/buildDirectory -S/path/to/srcDirectory
  make
\end{bash}

This should make a number of executables, e.g. {\tt ess}, {\tt maxiv},
{\tt reactor}, {\tt fullBuild} etc, which can be used to make a model
with commands listed in the {\tt all.sh} script. For instance, the
\begin{bash}
  singleItem --singleItem Octupole AA
\end{bash}
command generates an output file named {\tt AA1.x}, which is an MCNP
model of an octupole magnet. To generate a FLUKA model, add the {\tt
  -fluka} argument, and for PHITS models, use the {\tt -phits}
argument.
