
\section{Model Runtime control}

C++ programs start from the main() function and in CombLayer the
runtime control has been keep mostly in the main() function. Clearly
that could be further refactored out but CombLayer lacks the
sophisticated top level type abstraction that is required to do this
in a generic way, so copy/pasted structure is used with variance to
the particular model required. The sole advantage of the abscence of a
top level abstraction is that the user is the freedom in writing new objects
which allows other programs to be incorporated by making their main function
a minor function and directly calling.

The structure of two example main()s will be compared from the units
that exist with the stanard CombLayer distribution. That is {\it
bilbau.cxx} and {\it reactor.cxx}. These build the delft reactor model and the
Biblau low energy spallation source.

First part of the code is along list of \#include's. They are the
mainly dependency list of the objects {\it Simulation, weightManager,
and tallySelector.} This can and should be copied at will. Do not
make an file with them all in [see \ref{Sec:IntroInclude}].

At the end of the include section there is typically, one or two model
specific includes. These normally include {\it makeXXX.h} file and
anything that they directly depend on. In the case of bilbau it is just 
{\it makeBib.h} whilst for reactor it is both {\it makeDelft.h} and \prog{ReactorGrid.h}.



