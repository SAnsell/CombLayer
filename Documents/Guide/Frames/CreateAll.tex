\section{Expected Component Structure}
\label{Sec:Structure}

Definitions:

   An class that builds a placeable object: i.e a FixedComp or
   derivative. But not a componsite of objects that have their own
   offset/angles, even if they have a commong starting origin.

     E.g. if many object could have different placement to each other,
     then they are a composite (e.g. a beamline). But an object that
     has a few objects within it (for example bolts in a flange) and
     those included object don't have a variable modified origin relative
     to the primary origin. E.g they don't have a local xStep,yStep etc.


     If a class is a placeable object then it should inherrit from
     FixedComp directly or via one of the FixedComp derivatives in
     attachComp.

     It will not inherit from FixedUnit or any of its derivatives.



 Object:

   The object has to satify a number of abstract virtual methods.

     void createAll(Simulation&,const FixedComp&,const long int);

     This is a public method that builds the object. Typically it will call in order:

     populate(FuncDataBase);
     createUnitVector(FixedComp,const long int);
     createSurfaces();
     createObjects(System);
     createLinks();


     It is strongly discouraged to call any of these sub-methods with
     different arguments as normally this breaks the expected
     inheritance system. For example, if cutting surfaces are needed
     e.g. a front/back cut, then use ExternalCut or FrontBackCut, do
     not pass an FixedComp and link point to createSurfaces.


     It is also highly likely that for simple FixedComp, FixedRotate,
     and FixedOffset  objects, createUnitVector can be
     left out because these base classes provided methods
     are acceptable e.g.

     FixedComp::createUnitVector sets the Origin and basis set.

     FixedOffset::createUnitVector -- applied the pre and post rotations
     along with the offset movement set by the variables:

     Using preXYAngle,preZAngle,xStep,yStep,zStep,xyAngle,zAngle.

     FixedRotate::createUnitVector applies the pre- post rotations along
     all three axis


     Additional functionality of the reOrientate call is made after
     applying the rotations and offset. The function is automatically
     applied (if called for) in all of the FixedXXX class.

     The reOrientate function moves one axis to a another global axis with
     the minimum rotation

     Example ess model -- e.g. Vespa/Odin/
     
