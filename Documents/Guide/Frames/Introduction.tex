\section{Introduction}

CombLayer is designed to facilitate the rapid production of complex
MCNPX models that depend on a long list of ranged variables and a
number of module flags. It is also intended to help with placement of tallies,
maintaining consistant material files and some variance reduction.

\subsection{Coding Convensions}

CombLayer has some coding convensions beyond the standard Scott Meyers:
{\it Effective C++} conventions \cite{Meyers}. These are typcially their for
two reasons (i) the in a model-build system rapid build time is
essential since it is impossible to have a sub-test framework for any
component as the whole MCNPX model is required to check if is it
valid, (ii) the code is intended to be used without complete
understanding. Therefore as much as possible, each component is
independent without code repetition. Akk back-references are to be minimized
both in the run-time calling path and in the code build dependencies.

\subsubsection{Include files}
\label{Sec:IntroInclude}
 Include files (.h) are forbidden to include other
files. This does several things (a) it reduces the {\it dependency
  hell} where it is almost impossible to find the definition of a
function and what it depends on. (b) optimization of the include tree
can be carried out and dependency continuously observered. 

Namespaces are a good method of removing global name pollution but
many other C++ programs allows {\it using namespace X} this is almost
100\% forbidden except in the test for that particular namespace
unit. This also applies to boost, stl, tr1 etc, to which helps
distinguish external functions and domains.


